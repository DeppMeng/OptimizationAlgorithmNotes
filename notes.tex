% last updated in April 2002 by Antje Endemann
% Based on CVPR 07 and LNCS, with modifications by DAF, AZ and elle, 2008 and AA, 2010, and CC, 2011; TT, 2014; AAS, 2016

\documentclass[runningheads]{llncs}
\usepackage{graphicx}
\usepackage{amsmath,amssymb} % define this before the line numbering.
\usepackage{ruler}
\usepackage{color}
\usepackage[width=122mm,left=12mm,paperwidth=146mm,height=193mm,top=12mm,paperheight=217mm]{geometry}
\usepackage{multirow, multicol}
\usepackage[ruled,vlined]{algorithm2e}
    
% \newtheorem{theorem}{Theorem}
% \newtheorem{theorem}{Theorem}[section]

\begin{document}
% \renewcommand\thelinenumber{\color[rgb]{0.2,0.5,0.8}\normalfont\sffamily\scriptsize\arabic{linenumber}\color[rgb]{0,0,0}}
% \renewcommand\makeLineNumber {\hss\thelinenumber\ \hspace{6mm} \rlap{\hskip\textwidth\ \hspace{6.5mm}\thelinenumber}}
% \linenumbers
\pagestyle{headings}
\mainmatter
\def\ECCV18SubNumber{1452}  % Insert your submission number here

\title{Optimization Algorithm Notes} % Replace with your title

\titlerunning{ECCV-18 submission ID \ECCV18SubNumber}

\authorrunning{ECCV-18 submission ID \ECCV18SubNumber}

\author{Depu Meng}
\institute{Feb. $2019$}


\maketitle
\section{Introduction to Optimization Algorithms}
\subsection{Goal of the Course}
\begin{itemize}
    \item Understand foundations of optimization
    \item Learn to analyze widely used optimization algorithms
    \item Be familiar with implementation of optimization algorithms
\end{itemize}

\subsection{Topics Involved}
\begin{itemize}
    \item Unconstrained optimization
    \item Constrained optimization
    \item Convex optimization
    \item Sparse optimization
    \item Stochastic optimization
    \item Combinational optimization
    \item Global optimization
\end{itemize}

\subsection{Basic Concepts}
\noindent\textbf{Problem Definition}
Find the value of the decision variable s.t.
objective function is maximized/minimized
under certain conditions.
\begin{align}
    & \min f(x) \\
    & s.t. x \in \mathcal{S} \subset \mathbb{R}^n
\end{align}
Here, we call $\mathcal{S}$ \textit{feasible region}.
\par
We often denote constrained optimization Problem
 as 
\begin{align}
    & \min f(x) \\
    s.t.\quad & g_i(x) \geq 0, i = 1,..., n \\
    & b_i(x) = 0, i \in 1, ..., m
\end{align}

\begin{definition}{Global Optimality.}
    For global optimal value $x^* \in \mathcal{S}$,
    \begin{align}
        f(x^*) \leq f(x), \forall x \in \mathcal{S}
    \end{align}
\end{definition}

\begin{definition}{Local Optimality.}
    For local optimal value $x^* \in \mathcal{S}$, $\exists U(x^*)$,
    such that
    \begin{align}
        f(x^*) \leq f(x), \forall x \in \mathcal{S} \cap U(x^*)
    \end{align}
    \end{definition}

\begin{definition}{Feasible direction.}
    Let $x \in \mathcal{S}$, $d \in \mathbb{R}^n$
    is a non-zero vector. if $\exists \delta > 0$,
    such that
    \begin{align}
        x + \lambda d \in \mathcal{S}, \forall \lambda \in (0, \delta)
    \end{align}
    Then $d$ is a \textbf{feasible direction} at $x$.
    We denote $F(x, \mathcal{S})$ as the set of feasible directions
    at $x$.
\end{definition}

\begin{definition}{Descent direction.}
    $f(x): \mathbb{R}^n \rightarrow \mathbb{R}$, $x \in \mathbb{R}^n$,
    $d$ is a non-zero vector. If $\exists \delta > 0$, such that
    \begin{align}
        f(x + \lambda d) < f(x), \forall \lambda \in (0, \delta)
    \end{align}
    Then $d$ is a \textbf{descent direction} at $x$.
    We denote $D(x, f) = \{ d | \bigtriangledown f(x)^T d < 0 \}$
    as the set of descent direction at $x$.
\end{definition}

\subsection{Optimal Conditions}
\par\noindent
\textbf{Unconstrained Optimization}
\par
First-order necessery condition:
$f(x)$ is differentiable at $x$, 
\begin{align}
    \bigtriangledown f(x) = 0
\end{align}
\par
Second-order necessery condition:
$f(x)$ is second-order differentiable at $x$,
\begin{align}
    \bigtriangledown f(x) = 0 \\
    \bigtriangledown^2 f(x) \geq 0
\end{align}
\noindent
\textbf{Constrained Optimization}


\begin{theorem}{Fritz-John Condition}
    \par
    For constrained optimization problem
    \begin{align}
        & \min f(x) \\
        s.t.\quad & g_i(x) \geq 0, i = 1,..., n \\
        & h_i(x) = 0, i \in 1, ..., m
    \end{align}
    Denote $I(x) = \{i \in \{1,...,n\} | g_i(x) = 0\}$.
    For $x \in \mathcal{S}$, $f$ and $g_i, i \in I(x)$
    is differentiable at $x$, $h_j(x)$ is continuously
    differentiable at $x$.
    If $x$ is local optimal, then there exists non-trivial
    $\lambda_0, \lambda_i \geq 0, i \in I(x)$
    and $\mu_j$, such that
    \begin{align}
        \lambda_0 \bigtriangledown f(x) - \sum_{i \in I(x)} \lambda_i
        \bigtriangledown g_i(x) - \sum_{j=1}^m \mu_j
        \bigtriangledown h_j(x) = 0
    \end{align}
\end{theorem}

\begin{proof}
    (i) If $\{\bigtriangledown h_j (x)\}$ is linearly dependent,
    then there exists non-trivial $\mu_j$, such that
    \begin{align}
        \sum_{j=1}^m \bigtriangledown h_j (x) = 0
    \end{align}
    Let $\lambda_0, \lambda_i, i \in I(x) = 0$, then (13) holds.
    \par
    (ii) If $\{\bigtriangledown h_j (x)\}$ is linearly independent,
    Denote
    \begin{align}
        F_g &= F(x, g) = \{ d | \bigtriangledown g_i(x)^Td > 0, i \in I(x)\} \\
        F_h &= F(x, h) = \{ d | \bigtriangledown h_j(x)^Td = 0, j = 1,...,m\}
    \end{align}
    If $x$ is a optimal value, then appearently 
    $F(x, \mathcal{S}) \cap D(x, f) = \varnothing$.
    Due to the independence of $\{\bigtriangledown h_j (x)\}$,
    we have {\color{red}$F_g \cap F_h \subset F(x, \mathcal{S})$},
    then 
    \begin{align}
        F_g \cap F_h \cap D(x, f) = \varnothing
    \end{align}
    that is
    \begin{align}
        \left\{
        \begin{array}{ll}
            \bigtriangledown f(x)^Td < 0 \\
            \bigtriangledown g_i(x)^Td > 0, i \in I(x) \\
            \bigtriangledown h_j (x)^Td  =0, j = 1,...,m
        \end{array}
        \right.
    \end{align}
    has no solution.
    Let 
    \begin{align}
        A &= \{ \bigtriangledown f(x)^T, \bigtriangledown g_i(x)\}^T, i \in I(x) \\
        B &= \{ -\bigtriangledown h_j(x)\}, j = 1,...,m
    \end{align}
    Then (21) is equivalent to
    \begin{align}
        \left\{
        \begin{array}{ll}
            A^T d < 0 \\
            B^T d = 0
        \end{array}
        \right.
    \end{align}
    has no solution.
    \par
    \noindent
    Denote
    \begin{align}
        S_1 &= 
        \left\{
            \left(
            \begin{array}{ll}
                y_1 \\
                y_2
            \end{array}
            \right)
            | y_1 = A^Td, y_2 = B^Td, d \in \mathbb{R}^n
        \right\} \\
        S_2 &= 
        \left\{
            \left(
            \begin{array}{ll}
                y_1 \\
                y_2
            \end{array}
            \right)
            | y_1 < 0, y_2 = 0
        \right\} 
    \end{align}
    $S_1, S_2$ are non-trivial convex sets,
    and $S_1 \cap S_2 = \varnothing$.
    From \textit{Hyperplane Separation Theorem}:
    $\exists 
    \left(
    \begin{array}{ll}
        p_1 \\
        p_2
    \end{array}
    \right)$, so that
    \begin{align}
        p_1^TA^Td + p_2^TB^Td \geq p_1^Ty_1 + p_2^Ty_2,
        \forall d \in \mathbb{R}^n, 
        \forall\left(
        \begin{array}{ll}
            y_1 \\
            y_2
        \end{array}
        \right) \in CL(S_2)
    \end{align}
    Let $y_2 = 0, d = 0, y_1 < 0$, we have
    \begin{align}
        p_1 \geq 0
    \end{align}
    Let $\left(
        \begin{array}{ll}
            y_1 \\
            y_2
        \end{array}
        \right) = 
        \left(
        \begin{array}{ll}
            0 \\
            0
        \end{array}
        \right) 
         \in CL(S_2)$
    So that
    \begin{align}
        (p_1^TA^T + p_2^TB^T)d \geq 0 \\
        (Ap_1 + Bp_2)^Td \geq 0
    \end{align}
    Let $d = - (Ap_1 + Bp_2)$, we have
    \begin{align}
        Ap_1 + Bp_2 = 0
    \end{align}
    From above, we have
    \begin{align}
        \left\{
        \begin{array}{ll}
            Ap_1 + Bp_2 = 0 \\
            p_1 \geq 0
        \end{array}
        \right.
    \end{align}
    Let $p_1 = \{\lambda_0,..., \lambda_{I(x}\}$,
    $p_2 = \{\mu_1, ..., \mu_m\}$, i.e.,
    \begin{align}
        \left\{
            \begin{array}{ll}
            \lambda_0 \bigtriangledown f(x) - \sum_{i \in I(x)} \lambda_i
            \bigtriangledown g_i(x) - \sum_{j=1}^m \mu_j
            \bigtriangledown h_j(x) = 0 \\
            \lambda_i \geq 0
            \end{array}
            \right.
    \end{align}
\end{proof}

\begin{theorem}{Kuhn-Tucker Condition}
    \par
    For constrained optimization problem
    \begin{align}
        & \min f(x) \\
        s.t.\quad & g_i(x) \geq 0, i = 1,..., n \\
        & h_i(x) = 0, i \in 1, ..., m
    \end{align}
    Denote $I(x) = \{i \in \{1,...,n\} | g_i(x) = 0\}$.
    For $x \in \mathcal{S}$, $f$ and $g_i, i \in I(x)$
    is differentiable at $x$, $h_j(x)$ is continuously
    differentiable at $x$.
    $\{\bigtriangledown g_i(x), i \in I(x);\bigtriangledown h_j(x), 
    j = 1,...,m\}$ is linearly independent.
    If $x$ is local optimal, then $\exists \lambda_i \geq 0$
    and $\mu_j$, such that
    \begin{align}
        \bigtriangledown f(x) - \sum_{i \in I(x)} \lambda_i
        \bigtriangledown g_i(x) - \sum_{j=1}^m \mu_j
        \bigtriangledown h_j(x) = 0
    \end{align}
\end{theorem}
\subsection{Descent function}
\par
\begin{definition}{Descent function.}
    Denote solution set $\Omega \in X$, $\mathcal{A}$ is an
    algorithm on $X$, $\psi: X \rightarrow \mathbb{R}$.
    If
    \begin{align}
        \psi(y) < \psi(x),\quad \forall x \notin \Omega, y \in \mathcal{A}(x) \\
        \psi(y) \leq \psi(x),\quad \forall x \in \Omega, y \in \mathcal{A}(x)
    \end{align}
    Then $\psi$ is a \textbf{descent function} of $(\Omega, \mathcal{A})$.
\end{definition}
\subsection{Convergence of Algorithm}
\par
\begin{theorem}
    $\mathcal{A}$ is an algorithm on $X$,
    $\Omega$ is the solution set, $x^{(0)} \in X$.
    If $x^{(k)} \in \Omega$, then the iteration stops.
    Otherwise set $x^{(k+1)} = \mathcal{A}(x^{(k)}), k:=k+1$.
    If
    \begin{itemize}
        \item $\{x^{(k)}\}$ in a compact subset of $X$
        \item There exists a continuous function $\psi$,
         $\psi$ is a descent function of $(\Omega, \mathcal{A})$
        \item $\mathcal{A}$ is closed on $\Omega^C$
    \end{itemize}
    Then, any convergent subsequence of $\{x^{(k)}\}$ converges to 
    $x, x \in \Omega$.
\end{theorem}
\begin{proof}
    
\end{proof}

\subsection{Search Methods}
\noindent\textbf{Line Search}
\par
Generate $d^{(k)}$ from 
$x^{(k)}$, 
\begin{align}
    x^{(k+1)} = x^{(k)} + \alpha_kd^{(k)}
\end{align}.
search $\alpha_k$ in 1-D space.
\par
\noindent\textbf{Trust Region}
\par
Generate local model $Q_{k}(s)$ of $x^{(k)}$,
\begin{align}
    s^{(k)} = \mathop{\arg\min} Q_k(s) \\
    x^{(k+1)} = x^{(k)} + s^{(k)}
\end{align}

\clearpage
\section{Unconstrained Optimization}

\subsection{Gradient Based Methods}

\begin{algorithm}[H]
    \SetAlgoLined
    \KwData{Solution set $\Omega$, cost function $f$}
    % \KwResult{Write here the result }
     $x^{(0)} \in \mathbb{R}^n, k:= 0$\;
     \While{$x^{(k)} \notin \Omega$}{
      $d^{(k)} = - H_k \bigtriangledown f(x^{(k)})$,
        ($H_k$ is a positive definite symmetrical matrix)\;
        solve $\min_{\alpha_k \geq 0} f(x^{(k)} + \alpha_k d^{(k)})$\;
        $x^{(k+1)} = x^{(k)} + \alpha_kd^{(k)}$, $k:= k+1$
     }
     \caption{Example of gradient based algorithm}
\end{algorithm}
\subsection{Determine Search Direction}
\noindent\textbf{First-order gradient method}
\par
For unconstrained optimization problem 
\begin{align}
    \min_{x \in \mathbb{R}^n} f(x)
\end{align}
We have
\begin{align}
    f(x) = f(x^{(k)}) + \bigtriangledown f(x^{(k)})^T(x - x^{(k)})
    + O(\parallel x - x^{(k)} \parallel^2)
\end{align}
Set $d^{(k)} = - \bigtriangledown f(x^{(k)})$,
when $\alpha_k$ is sufficiently small,
\begin{align}
    f(x^{(k)} + \alpha_k d^{(k)}) < f(x^{(k)})
\end{align}

\noindent\textbf{Second-order gradient method -- Newton Direction}
\par
\begin{align}
    f(x) = &f(x^{(k)}) + \bigtriangledown f(x^{(k)})^T(x - x^{(k)}) \\
    &+ \frac{1}{2} (x - x^{(k)})^T \bigtriangledown^2 f(x^{(k)}) (x - x^{(k)})
    + O(\parallel x - x^{(k)} \parallel^3)
\end{align}
Set $d^{(k)} = -G_k^{-1} \bigtriangledown f(x^{(k)})$,
 where $G_k = \bigtriangledown^2 f(x^{(k)})$,
 i.e., Hesse matrix of $f$ at $x^{(k)}$.
 \subsection{Determine Step Factor}

\end{document}