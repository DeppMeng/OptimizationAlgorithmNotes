\chapter{Convex Optimization}
\vspace{1em}
\section{Convex set}
\begin{definition}[Convex Set]
    A set $\mathcal{C} \subset \mathbb{R}^n$ affine
    if $\mathbf{x}_1, \mathbf{x}_2 \in \mathcal{C}$
    and $\theta \in \mathbb{R}$, we have
    \begin{align}
        \theta \mathbf{x}_1 + 
        (1 - \theta)\mathbf{x}_2
        \in \mathcal{C}
    \end{align}
\end{definition}
\section{Convex function}
\section{Convex optimization}
\subsection{Quadratically constrained quadratic program}
\begin{align}
    \begin{array}{lll}
        \min \ &\frac{1}{2} \mathbf{x}^T P_0 \mathbf{x}
        + \mathbf{q}_0^T \mathbf{x} + r_0 \\
        s.t. \ &\frac{1}{2} \mathbf{x}^T P_i \mathbf{x}
        + \mathbf{q}_i^T \mathbf{x} + r_i \leq 0, \quad i = 1,...,m \\
        & A \mathbf{x} = \mathbf{b}
    \end{array}
\end{align}


\par
\subsection{Second-order cone program}
\begin{align}
    \begin{array}{lll}
        \min \ & \mathbf{f}^T \mathbf{x} \\
        s.t. \ & \parallel A_i \mathbf{x} + \mathbf{b}_i \parallel
        \leq \mathbf{c}_i^T\mathbf{x} + \mathbf{d}_i, \quad i = 1,...,m \\
        & F\mathbf{x} = \mathbf{g}
    \end{array}
\end{align}

\begin{lemma}
    Any QCQP problem can be formulated as a SOCP problem.
\end{lemma}
\begin{proof}
    The QCQP problem is equivalent to
    \begin{align}
        \min \ & - r_0 \\
        s.t. \ &\frac{1}{2} \mathbf{x}^T P_i \mathbf{x}
        + \mathbf{q}_i^T \mathbf{x} + r_i \leq 0, \quad i = 0,...,m \\
        & A \mathbf{x} = \mathbf{b}
    \end{align}
    Then we need to prove that (121) can be formulated as (118).
    \begin{align}
        &\frac{1}{2} \mathbf{x}^T P_i \mathbf{x}
        + \mathbf{q}_i^T \mathbf{x} + r_i \leq 0 \\
        \Leftrightarrow \ & \mathbf{x}^T P_i \mathbf{x}
        + 2(\mathbf{q}_i^T \mathbf{x} + r_i) \leq 0 \\
        \Leftrightarrow \ & \mathbf{x}^T P_i \mathbf{x}
        + 2(\mathbf{q}_i^T \mathbf{x} + r_i)
        + (\mathbf{q}_i^T \mathbf{x} + r_i - \frac{1}{2})^2
         \leq (\mathbf{q}_i^T \mathbf{x} + r_i - \frac{1}{2})^2 \\
         \Leftrightarrow \ & \mathbf{x}^T P_i \mathbf{x}
         + (\mathbf{q}_i^T \mathbf{x} + r_i + \frac{1}{2})^2
          \leq (\mathbf{q}_i^T \mathbf{x} + r_i - \frac{1}{2})^2
    \end{align}
    Since $P_i$ is positive semi-definite, $P_i = A_i^TA_i$, then
    \begin{align}
        \Leftrightarrow \ & \mathbf{x}^T P_i \mathbf{x}
        + (\mathbf{q}_i^T \mathbf{x} + r_i + \frac{1}{2})^2
         \leq (\mathbf{q}_i^T \mathbf{x} + r_i - \frac{1}{2})^2 \\
         \Leftrightarrow \ & \parallel A_i \mathbf{x} \parallel^2
         + \parallel \mathbf{q}_i^T \mathbf{x} + r_i + \frac{1}{2}\parallel^2
          \leq (\mathbf{q}_i^T \mathbf{x} + r_i - \frac{1}{2})^2 
    \end{align}
    Let
    \begin{align}
        A_i' &= \left(
            \begin{array}{ll}
                A \\
                \mathbf{q}^T
            \end{array}\right) \\
            \mathbf{b}_i &= \left(
                \begin{array}{ll}
                    \mathbf{0}_{n\times1} \\
                    r_i + \frac{1}{2}
                \end{array}\right)
    \end{align}
    From (123) and $\mathbf{x}^T P_i \mathbf{x} \geq 0$,
    we can derive that $\mathbf{q}_i^T \mathbf{x} + r_i \leq 0$,
    i.e., $\mathbf{q}_i^T \mathbf{x} + r_i - \frac{1}{2} \leq 0$.
    \par
    Then (128) can be formulated as
    \begin{align}
        &\parallel A_i'\mathbf{x} + \mathbf{b}_i \parallel^2
        \leq (\mathbf{q}_i^T \mathbf{x} + r_i - \frac{1}{2})^2 \\
        \Leftrightarrow \ & \parallel A_i'\mathbf{x} + \mathbf{b}_i \parallel
        \leq -(\mathbf{q}_i^T \mathbf{x} + r_i - \frac{1}{2})
    \end{align}
\end{proof}