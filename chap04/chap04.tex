\chapter{Convex Optimization}
\vspace{1em}
\section{Convex set}
\subsection{Affine set}
\begin{definition}[Affine set]
    A set $\mathcal{C} \subset \mathbb{R}^n$ is affine
    if $\mathbf{x}_1, \mathbf{x}_2 \in \mathcal{C}$
    and $\theta \in \mathbb{R}$, we have
    \begin{align}
        \theta \mathbf{x}_1 + 
        (1 - \theta)\mathbf{x}_2
        \in \mathcal{C}
    \end{align}
\end{definition}

\begin{definition}[Affine hull]
    The set of all affine combinations of points
    in some set $\mathcal{C} \subset \mathbb{R}^n$
    is called the \emph{affine hull} of $\mathcal{C}$,
    denoted $\mathbf{aff}\mathcal{C}$:
    \begin{align}
        \mathbf{aff}\mathcal{C} = \{
            \sum_{i=1}^k \theta_i \mathbf{x}_i \ | \ 
            \mathbf{x}_1,...,\mathbf{x}_k \in \mathcal{C},
            \theta_1 + ... + \theta_k = 1\}
    \end{align}
\end{definition}
\begin{remark}
    The affine hull is the smallest affine set
    that contains $\mathcal{C}$.
\end{remark}
\begin{proof}
    For any affine set $\mathcal{A}$ contains $\mathcal{C}$,
    we have
    \begin{align}
        \sum_{i=1}^k \theta_i \mathbf{x}_i \in \mathcal{A},
        \forall \mathbf{x}_1,...,\mathbf{x}_k \in \mathcal{C},
            \theta_1 + ... + \theta_k = 1
    \end{align}
    i.e., $\mathbf{aff}\mathcal{C} \subset \mathcal{A}$.
\end{proof}

\subsection{Convex set}
\begin{definition}[Convex set]
    A set $\mathcal{C} \subset \mathbb{R}^n$ is convex
    if $\mathbf{x}_1, \mathbf{x}_2 \in \mathcal{C}$
    and $0 \leq \theta \leq 1$, we have
    \begin{align}
        \theta \mathbf{x}_1 + 
        (1 - \theta)\mathbf{x}_2
        \in \mathcal{C}
    \end{align}
\end{definition}
\begin{definition}[Convex hull]
    The set of all convex combinations of points
    in some set $\mathcal{C} \subset \mathbb{R}^n$
    is called the \emph{convex hull} of $\mathcal{C}$,
    denoted $\mathbf{conv}\mathcal{C}$:
    \begin{align}
        \mathbf{conv}\mathcal{C} = \{
            \sum_{i=1}^k \theta_i \mathbf{x}_i \ | \ 
            \mathbf{x}_1,...,\mathbf{x}_k \in \mathcal{C},
            \theta_i \geq 0,
            \theta_1 + ... + \theta_k = 1\}
    \end{align}
\end{definition}
\begin{remark}
    The convex hull is the smallest convex set
    that contains $\mathcal{C}$.
\end{remark}

\subsection{Cone}
\begin{definition}[Cone]
    A set $\mathcal{C}$ is called a \emph{cone},
    if $\forall \mathbf{x} \in \mathcal{C}$ and
    $\theta \geq 0$ we have $\theta \mathbf{x} \in \mathcal{C}$.
    A set $\mathcal{C}$ is called a \emph{convex cone}
    if it is convex and a cone, i.e.,
    $\forall \mathbf{x}_1, \mathbf{x})_2 \in \mathcal{C}$
    and $\theta_1, \theta_2 \geq 0$, we have
    \begin{align}
        \theta_1 \mathbf{x}_1 + \theta_2 \mathbf{x}_2
        \in \mathcal{C}
    \end{align}
\end{definition}

\begin{definition}[Conic hull]
    The conic hull of set $\mathcal{C}$ is the
    set of all conic combinations of points in
    $\mathcal{C}$, i.e.,
    \begin{align}
        \{ \sum_{i=1}^k \theta_i \mathbf{x}_i \ | \
        \mathbf{x}_i \in \mathcal{C}, \theta_i \geq 0,
        i = 1,...,k \}
    \end{align}
\end{definition}

\subsection{Proper cones and generalized inequalities}



\section{Convex function}
\begin{definition}[Convex function]
    A function $f:\mathbb{R}^n \rightarrow \mathbb{R}$
    is \emph{convex} if $\mathbf{dom}f$ is a convex
    set and if $\forall x, y \in \mathbf{dom}f$ and
    $\theta$ with $0 \leq \theta\leq 1$, we have
    \begin{align}
        f(\theta x_1 + (1 - \theta) x_2) \leq
        \theta f(x_1) + (1 - \theta) f(x_2)
    \end{align}
\end{definition}

\subsection{First order condition}
Suppose $f$ is differentiable
\begin{theorem}
    Function $f$ is convex if and only if $\mathbf{dom}f$
    is a convex set and for $\forall x, y \in \mathbf{dom}f$,
    the following holds:
    \begin{align}
        f(y) \geq f(x) + \bigtriangledown f(x)^T (y - x)
    \end{align}
\end{theorem}

\begin{remark}
    If $\bigtriangledown f(x^*) = 0$, then for
    $\forall y \in \mathbf{dom}f$, 
    $f(y) \geq f(x^*)$, i.e., $x^*$
    is the global minimizer of $f$.
\end{remark}


\section{Convex optimization}
\subsection{Quadratically constrained quadratic program}
\begin{align}
    \begin{array}{lll}
        \min \ &\frac{1}{2} \mathbf{x}^T P_0 \mathbf{x}
        + \mathbf{q}_0^T \mathbf{x} + r_0 \\
        s.t. \ &\frac{1}{2} \mathbf{x}^T P_i \mathbf{x}
        + \mathbf{q}_i^T \mathbf{x} + r_i \leq 0, \quad i = 1,...,m \\
        & A \mathbf{x} = \mathbf{b}
    \end{array}
\end{align}


\par
\subsection{Second-order cone program}
\begin{align}
    \begin{array}{lll}
        \min \ & \mathbf{f}^T \mathbf{x} \\
        s.t. \ & \parallel A_i \mathbf{x} + \mathbf{b}_i \parallel
        \leq \mathbf{c}_i^T\mathbf{x} + \mathbf{d}_i, \quad i = 1,...,m \\
        & F\mathbf{x} = \mathbf{g}
    \end{array}
\end{align}

\begin{lemma}
    Any QCQP problem can be formulated as a SOCP problem.
\end{lemma}
\begin{proof}
    The QCQP problem is equivalent to
    \begin{align}
        \min \ & - r_0 \\
        s.t. \ &\frac{1}{2} \mathbf{x}^T P_i \mathbf{x}
        + \mathbf{q}_i^T \mathbf{x} + r_i \leq 0, \quad i = 0,...,m \\
        & A \mathbf{x} = \mathbf{b}
    \end{align}
    Then we need to prove that (121) can be formulated as (118).
    \begin{align}
        &\frac{1}{2} \mathbf{x}^T P_i \mathbf{x}
        + \mathbf{q}_i^T \mathbf{x} + r_i \leq 0 \\
        \Leftrightarrow \ & \mathbf{x}^T P_i \mathbf{x}
        + 2(\mathbf{q}_i^T \mathbf{x} + r_i) \leq 0 \\
        \Leftrightarrow \ & \mathbf{x}^T P_i \mathbf{x}
        + 2(\mathbf{q}_i^T \mathbf{x} + r_i)
        + (\mathbf{q}_i^T \mathbf{x} + r_i - \frac{1}{2})^2
         \leq (\mathbf{q}_i^T \mathbf{x} + r_i - \frac{1}{2})^2 \\
         \Leftrightarrow \ & \mathbf{x}^T P_i \mathbf{x}
         + (\mathbf{q}_i^T \mathbf{x} + r_i + \frac{1}{2})^2
          \leq (\mathbf{q}_i^T \mathbf{x} + r_i - \frac{1}{2})^2
    \end{align}
    Since $P_i$ is positive semi-definite, $P_i = A_i^TA_i$, then
    \begin{align}
        \Leftrightarrow \ & \mathbf{x}^T P_i \mathbf{x}
        + (\mathbf{q}_i^T \mathbf{x} + r_i + \frac{1}{2})^2
         \leq (\mathbf{q}_i^T \mathbf{x} + r_i - \frac{1}{2})^2 \\
         \Leftrightarrow \ & \parallel A_i \mathbf{x} \parallel^2
         + \parallel \mathbf{q}_i^T \mathbf{x} + r_i + \frac{1}{2}\parallel^2
          \leq (\mathbf{q}_i^T \mathbf{x} + r_i - \frac{1}{2})^2 
    \end{align}
    Let
    \begin{align}
        A_i' &= \left(
            \begin{array}{ll}
                A \\
                \mathbf{q}^T
            \end{array}\right) \\
            \mathbf{b}_i &= \left(
                \begin{array}{ll}
                    \mathbf{0}_{n\times1} \\
                    r_i + \frac{1}{2}
                \end{array}\right)
    \end{align}
    From (123) and $\mathbf{x}^T P_i \mathbf{x} \geq 0$,
    we can derive that $\mathbf{q}_i^T \mathbf{x} + r_i \leq 0$,
    i.e., $\mathbf{q}_i^T \mathbf{x} + r_i - \frac{1}{2} \leq 0$.
    \par
    Then (128) can be formulated as
    \begin{align}
        &\parallel A_i'\mathbf{x} + \mathbf{b}_i \parallel^2
        \leq (\mathbf{q}_i^T \mathbf{x} + r_i - \frac{1}{2})^2 \\
        \Leftrightarrow \ & \parallel A_i'\mathbf{x} + \mathbf{b}_i \parallel
        \leq -(\mathbf{q}_i^T \mathbf{x} + r_i - \frac{1}{2})
    \end{align}
\end{proof}